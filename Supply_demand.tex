\documentclass[a4paper]{article}\usepackage[]{graphicx}\usepackage[]{color}
%% maxwidth is the original width if it is less than linewidth
%% otherwise use linewidth (to make sure the graphics do not exceed the margin)
\makeatletter
\def\maxwidth{ %
  \ifdim\Gin@nat@width>\linewidth
    \linewidth
  \else
    \Gin@nat@width
  \fi
}
\makeatother

\definecolor{fgcolor}{rgb}{0.345, 0.345, 0.345}
\newcommand{\hlnum}[1]{\textcolor[rgb]{0.686,0.059,0.569}{#1}}%
\newcommand{\hlstr}[1]{\textcolor[rgb]{0.192,0.494,0.8}{#1}}%
\newcommand{\hlcom}[1]{\textcolor[rgb]{0.678,0.584,0.686}{\textit{#1}}}%
\newcommand{\hlopt}[1]{\textcolor[rgb]{0,0,0}{#1}}%
\newcommand{\hlstd}[1]{\textcolor[rgb]{0.345,0.345,0.345}{#1}}%
\newcommand{\hlkwa}[1]{\textcolor[rgb]{0.161,0.373,0.58}{\textbf{#1}}}%
\newcommand{\hlkwb}[1]{\textcolor[rgb]{0.69,0.353,0.396}{#1}}%
\newcommand{\hlkwc}[1]{\textcolor[rgb]{0.333,0.667,0.333}{#1}}%
\newcommand{\hlkwd}[1]{\textcolor[rgb]{0.737,0.353,0.396}{\textbf{#1}}}%
\let\hlipl\hlkwb

\usepackage{framed}
\makeatletter
\newenvironment{kframe}{%
 \def\at@end@of@kframe{}%
 \ifinner\ifhmode%
  \def\at@end@of@kframe{\end{minipage}}%
  \begin{minipage}{\columnwidth}%
 \fi\fi%
 \def\FrameCommand##1{\hskip\@totalleftmargin \hskip-\fboxsep
 \colorbox{shadecolor}{##1}\hskip-\fboxsep
     % There is no \\@totalrightmargin, so:
     \hskip-\linewidth \hskip-\@totalleftmargin \hskip\columnwidth}%
 \MakeFramed {\advance\hsize-\width
   \@totalleftmargin\z@ \linewidth\hsize
   \@setminipage}}%
 {\par\unskip\endMakeFramed%
 \at@end@of@kframe}
\makeatother

\definecolor{shadecolor}{rgb}{.97, .97, .97}
\definecolor{messagecolor}{rgb}{0, 0, 0}
\definecolor{warningcolor}{rgb}{1, 0, 1}
\definecolor{errorcolor}{rgb}{1, 0, 0}
\newenvironment{knitrout}{}{} % an empty environment to be redefined in TeX

\usepackage{alltt}
\usepackage[cm]{fullpage}
\usepackage{graphicx}
\usepackage{float}
\usepackage{caption}
\textwidth=18cm
\usepackage[backend=bibtex]{biblatex}



\IfFileExists{upquote.sty}{\usepackage{upquote}}{}
\begin{document}
\title{Sperry's supply-demand-loss model}
\author{Liam Langan}
\maketitle


\section{Introduction}
Sperry and Love (2015 (What plant hydraulics can tell us about responses to climate-change droughts)) developed a model where a supply function (transpiration rate) ($E$) (kg hr$^{-1}$ m$^{-2}$) is derived which calculates the potential rate/amount of water able to be supplied from the soil to the atmosphere. Transpiration ($E$) is influenced by the canopy sap pressure (p\_canopy) (MPa) via changes to the hydraulic conductance of the plant ($k$) (kg hr$^{-1}$ MPa$^{-1}$ m$^{-2}$). Hydraulic conductance of the plant ($k$) is the conductance when there is no difference in matric potential between the soil and the leaf.  

Below are parameters for the vulnerability-conductance curve:
\begin{knitrout}
\definecolor{shadecolor}{rgb}{0.969, 0.969, 0.969}\color{fgcolor}\begin{kframe}
\begin{alltt}
\hlcom{##-----------------------------------------------------------------------------------------------}
\hlstd{p50} \hlkwb{<-} \hlnum{2.5} \hlcom{# the matric potential where conductance is reduced by 50% (trait aDGVM2)}
\hlstd{k_max} \hlkwb{<-} \hlnum{8} \hlcom{# maximum plant conductance - (fn of p50)}
\hlstd{res} \hlkwb{<-} \hlnum{1}\hlopt{/}\hlstd{k_max} \hlcom{# resistance is simply the inverse of conductance}
\hlstd{p_canopy} \hlkwb{<-} \hlkwd{matrix}\hlstd{(}\hlkwd{seq}\hlstd{(}\hlnum{0.0}\hlstd{,} \hlnum{8}\hlstd{,} \hlkwc{length}\hlstd{=}\hlnum{1000}\hlstd{),} \hlkwd{seq}\hlstd{(}\hlnum{0.0}\hlstd{,} \hlnum{8}\hlstd{,} \hlkwc{length}\hlstd{=}\hlnum{1000}\hlstd{),} \hlkwc{nrow}\hlstd{=}\hlnum{1000}\hlstd{,} \hlkwc{ncol}\hlstd{=}\hlnum{1000}\hlstd{)}
\hlcom{#p_canopy <- seq(0.0, 8, length=1000)}
\hlcom{# this assumes initial plant matric potential is the same as the soil matric potential}
\hlstd{predawn_soil_mat_pot} \hlkwb{<-} \hlkwd{seq}\hlstd{(}\hlnum{0}\hlstd{,}\hlnum{8}\hlstd{,} \hlkwc{length}\hlstd{=}\hlnum{1000}\hlstd{)}
\hlstd{E_p_canopy} \hlkwb{<-} \hlkwd{matrix}\hlstd{(}\hlnum{0}\hlstd{,}\hlnum{0}\hlstd{,}\hlkwc{nrow}\hlstd{=}\hlnum{1000}\hlstd{,} \hlkwc{ncol}\hlstd{=}\hlnum{1000}\hlstd{)} \hlcom{# matrix to hold the supply function values}
\end{alltt}
\end{kframe}
\end{knitrout}
\noindent with the the conductance vulnerability curve ($k\_p\_canopy$) we use in aDGVM2, which is analagous to Sperry's curve, defined as:
\begin{knitrout}
\definecolor{shadecolor}{rgb}{0.969, 0.969, 0.969}\color{fgcolor}\begin{kframe}
\begin{alltt}
\hlstd{k_p_canopy} \hlkwb{<-} \hlkwa{function}\hlstd{(}\hlkwc{p_canopy}\hlstd{) \{ ((}\hlnum{1} \hlopt{-} \hlstd{(}\hlnum{1} \hlopt{/} \hlstd{(}\hlnum{1} \hlopt{+} \hlkwd{exp}\hlstd{(}\hlnum{3.0}\hlopt{*}\hlstd{(p50} \hlopt{-} \hlstd{p_canopy))))))} \hlopt{/} \hlstd{res \}}
\end{alltt}
\end{kframe}
\end{knitrout}
\noindent We set the maximum stomatal conductance $Gmax$ to 12563.1 (kg h$^{-1}$ MPa$^{-1}$ m$^{-2}$) based on the value used by Sperry (2016, Excel table). The transpiration demand $E1$ is equal to 

\begin{equation}
E1 = Gmax * D
\end{equation}

\noindent where $D$ (kPa) is the leaf to air vapor pressure defecit. The matric potential where, once passed, it is assumed that runaway cavitation is the result ($P\_crit$) is arbitrary but is a point on the ($k\_p\_canopy$) curve where the slope of the tangent to the curve is very close to zero. The maximum transpiration rate beyond which runaway cavitation is assumed ($E\_crit$). This is the transpiraiton rate at $P\_crit$.   
\begin{knitrout}
\definecolor{shadecolor}{rgb}{0.969, 0.969, 0.969}\color{fgcolor}\begin{kframe}
\begin{alltt}
\hlcom{##-----------------------------------------------------------------------------------------------}
\hlcom{# Maximum stomatal conductance}
\hlcom{# (Sperry 2016, 2130 kg h^-1 m^-2) NOTE should be (kg h^-1 MPa^-1 m^-2) (12563.1 in Excel doc)}
\hlstd{Gmax} \hlkwb{<-} \hlnum{12563.1}
\hlstd{D} \hlkwb{<-} \hlnum{1.0}\hlopt{*}\hlnum{0.001} \hlcom{#(Sperry 2016, leaf-to-air vapor pressure deficit 1 kPa)(0.001 converts to MPa) }
\hlcom{# NOTE VPD converstion from kPa to MPa isn't documented in Sperry, I'm doing it as it makes }
\hlcom{# sense and produces realistic amounts of transpirtational demand. }

\hlcom{# Pcrit, i.e. a matric potential we choose where we decide conductance is effectively zero. }
\hlcom{# Used this to get Ecrit, i.e. maximum transpiration beyond which leads to runaway cavitation  }
\hlstd{P_crit} \hlkwb{<-} \hlnum{6} \hlcom{# MPa - this is arbitrary and could be a plant trait. }
\hlcom{# In Sperry (2016) a P_crit cutoff is chosen (either very low conductance or }
\hlcom{# shallow slope of a tangent to the transpiration curve)}

\hlcom{# get maximum transpiration possible based on Pcrit and soil matric potential }
\hlstd{E_crit} \hlkwb{<-} \hlkwd{rep}\hlstd{(}\hlnum{0}\hlstd{,} \hlkwc{lenght}\hlstd{=}\hlnum{1000}\hlstd{)} \hlcom{# maximum transpiration beyond which leads to runaway cavitation}
\hlstd{E1} \hlkwb{<-} \hlkwd{rep}\hlstd{(}\hlnum{0}\hlstd{,} \hlkwc{lenght}\hlstd{=}\hlnum{1000}\hlstd{)} \hlcom{# evaporative demand}
\hlstd{Max_gs_test} \hlkwb{<-} \hlkwd{rep}\hlstd{(}\hlnum{0}\hlstd{,}\hlkwc{length}\hlstd{=}\hlnum{1000}\hlstd{)}
\end{alltt}
\end{kframe}
\end{knitrout}

\noindent $E\_crit$ is calculated as the integral of ($k\_p\_canopy$) between the predawn soil matric potential and $P\_crit$. If the transpirational demand $E1$ is greater than the maximum transpiration rate $E\_crit$, then demand is set to the maximum supply. 
\begin{knitrout}
\definecolor{shadecolor}{rgb}{0.969, 0.969, 0.969}\color{fgcolor}\begin{kframe}
\begin{alltt}
\hlkwa{for}\hlstd{(j} \hlkwa{in} \hlnum{1}\hlopt{:}\hlnum{1000}\hlstd{)}
\hlstd{\{}
  \hlstd{ffy} \hlkwb{<-} \hlkwd{integrate}\hlstd{(k_p_canopy, predawn_soil_mat_pot[j], P_crit )} \hlcom{# supply up to P_crit}
  \hlstd{E_crit[j]} \hlkwb{<-} \hlkwd{pmax}\hlstd{(}\hlnum{0}\hlstd{, ffy}\hlopt{$}\hlstd{value)}
  \hlstd{E1[j]} \hlkwb{<-} \hlstd{Gmax}\hlopt{*}\hlstd{D}
  \hlkwa{if}\hlstd{(E1[j]} \hlopt{>} \hlstd{E_crit[j]) E1[j]} \hlkwb{<-} \hlstd{E_crit[j]} \hlcom{# demand = maximum supply}
  \hlstd{Max_gs_test[j]} \hlkwb{<-} \hlstd{E1[j]}\hlopt{/}\hlstd{D}
\hlstd{\}}
\end{alltt}
\end{kframe}
\end{knitrout}


The following code calculates the maximum slope of the supply function from ($k\_p\_canopy$) across a range of predawn soil matric potentials. The slope of the supply function ($slope\_supply$) is calculated between for a range of between the predawn matric potentials and $p\_canopy$. The supply function ($E\_p\_canopy$) is the integral between the predawn soil matric potential and p\_canopy, this is performed for 1000 levels of predawn matric potential and p\_canopy, both (0-8 MPa). Non regulated stomatal/diffusive conductance ($non_regulated_Gs$) is simply the supply function divided by the vapor pressure defecit $D$. Sperry's loss function (loss\_fun\_sp) is simply the slope of the tangent to the supply curve line (which is the value of $k\_p\_canopy$) at any particular matric potential (slope\_supply), divided by the maximum slope (max\_slo\_sp). In Sperry's model this is at the predawn matric potential (predawn\_soil\_mat\_pot). The amount by which leaf pressure is adjusted ($delta\_P$) is calcualted as the difference between $p\_canopy$ and the predawn soil matric potential times the loss function. ($delta\_P$) is held at its maximum value once the maximum value is passed, this is the trick I was missing when Sperry writes about $delta\_P$ saturating. Regulated transpiration or a regulated supply function $regulated\_trans$ is calcualted as the integral of $k\_p\_canopy$ between the predawn matric potential and the predawn matric potential plus $delta\_P$. Regulated stomatal/diffusive conductance is then calcualted by dividing $regulated\_trans$ by the vapor pressure defecit $D$.      

\begin{knitrout}
\definecolor{shadecolor}{rgb}{0.969, 0.969, 0.969}\color{fgcolor}\begin{kframe}
\begin{alltt}
\hlkwa{for}\hlstd{(j} \hlkwa{in} \hlnum{1}\hlopt{:}\hlnum{1000}\hlstd{)}
\hlstd{\{}

  \hlstd{max_slo_sp[j]} \hlkwb{<-} \hlkwd{k_p_canopy}\hlstd{(predawn_soil_mat_pot[j])}

  \hlkwa{for}\hlstd{(i} \hlkwa{in} \hlnum{1}\hlopt{:}\hlnum{1000}\hlstd{)}
  \hlstd{\{}
   \hlstd{slope_supply[i,j]} \hlkwb{<-} \hlkwd{k_p_canopy}\hlstd{(}\hlkwd{pmax}\hlstd{(predawn_soil_mat_pot[j], p_canopy[i]))}

   \hlcom{# half of these ingegrations are not necessary as predawn>=p_canopy}
   \hlstd{ffx} \hlkwb{<-} \hlkwd{integrate}\hlstd{(k_p_canopy, predawn_soil_mat_pot[j], p_canopy[i] )}
   \hlstd{E_p_canopy[i,j]} \hlkwb{<-} \hlkwd{pmax}\hlstd{(}\hlnum{0}\hlstd{, ffx}\hlopt{$}\hlstd{value)}
   \hlstd{non_regulated_Gs[i,j]} \hlkwb{<-} \hlstd{E_p_canopy[i,j]}\hlopt{/}\hlstd{D}
   \hlcom{# E = G*VPD ---- G = E/VPD (VPD=1, 0.001 transforms to MPa)}

   \hlstd{loss_fun_sp[i,j]} \hlkwb{<-} \hlstd{slope_supply[i,j]} \hlopt{/} \hlstd{max_slo_sp[j]}

   \hlkwa{if}\hlstd{(i}\hlopt{==}\hlnum{1}\hlstd{) delta_P[i,j]} \hlkwb{<-} \hlkwd{pmax}\hlstd{(}\hlnum{0}\hlstd{, ((p_canopy[i]} \hlopt{-} \hlstd{predawn_soil_mat_pot[j])}
                                          \hlopt{*}\hlstd{loss_fun_sp[i,j]))}

   \hlkwa{if}\hlstd{(i}\hlopt{>}\hlnum{1}\hlstd{)}
   \hlstd{\{}
\hlcom{#max regulation is the point where delta P hits its maximum, held constant at max once max passed}
     \hlstd{delta_P[i,j]} \hlkwb{<-} \hlkwd{pmax}\hlstd{(}\hlnum{0}\hlstd{,} \hlkwd{pmax}\hlstd{(delta_P[i}\hlopt{-}\hlnum{1}\hlstd{,j],}
                              \hlstd{((p_canopy[i]} \hlopt{-} \hlstd{predawn_soil_mat_pot[j])}\hlopt{*}\hlstd{loss_fun_sp[i,j])))}
   \hlstd{\}}

   \hlcom{# half of these ingegrations are not necessary as delta_P will be zero}
   \hlstd{ffx} \hlkwb{<-} \hlkwd{integrate}\hlstd{(k_p_canopy, predawn_soil_mat_pot[j],}
                    \hlstd{predawn_soil_mat_pot[j]} \hlopt{+} \hlstd{delta_P[i,j])}
   \hlstd{regulated_trans[i,j]} \hlkwb{<-} \hlkwd{pmax}\hlstd{(}\hlnum{0}\hlstd{, ffx}\hlopt{$}\hlstd{value)}
   \hlstd{regulated_Gs[i,j]} \hlkwb{<-} \hlstd{regulated_trans[i,j]}\hlopt{/}\hlstd{D} \hlcom{# E=G*D -- G=E/D (D=1, 0.001 transforms to MPa)}
\hlcom{#   G[i,j] <- G[i,j]*loss_fun_sp[i,j]}
  \hlstd{\}}
\hlstd{\}}
\end{alltt}
\end{kframe}
\end{knitrout}
\noindent The below code finds the the first position in the $E\_p\_canopy$ dataset where supply is greater than or equal to demand.  
\begin{knitrout}
\definecolor{shadecolor}{rgb}{0.969, 0.969, 0.969}\color{fgcolor}\begin{kframe}
\begin{alltt}
\hlstd{demand_place_holder} \hlkwb{<-} \hlkwd{rep}\hlstd{(}\hlnum{0}\hlstd{,} \hlkwc{length}\hlstd{=}\hlnum{1000}\hlstd{)}
\hlstd{supply_limit_place_holder} \hlkwb{<-} \hlkwd{rep}\hlstd{(}\hlnum{0}\hlstd{,} \hlkwc{length}\hlstd{=}\hlnum{1000}\hlstd{)}
\hlstd{min_diff_place_holder} \hlkwb{<-} \hlkwd{rep}\hlstd{(}\hlnum{0}\hlstd{,} \hlkwc{length}\hlstd{=}\hlnum{1000}\hlstd{)}
\hlstd{supply_place_holder} \hlkwb{<-} \hlkwd{rep}\hlstd{(}\hlnum{0}\hlstd{,} \hlkwc{length}\hlstd{=}\hlnum{1000}\hlstd{)}

\hlkwa{for}\hlstd{(i} \hlkwa{in} \hlnum{1}\hlopt{:}\hlnum{1000}\hlstd{)}
\hlstd{\{}
    \hlstd{a} \hlkwb{<-}\hlkwd{which}\hlstd{(E_p_canopy[,i]} \hlopt{>=} \hlstd{E1[i])}
    \hlstd{demand_place_holder[i]} \hlkwb{<-}\hlstd{a[}\hlnum{2}\hlstd{]}

    \hlstd{min_diff} \hlkwb{<-} \hlstd{regulated_trans[,i]} \hlopt{-} \hlstd{E_p_canopy[demand_place_holder[i]]}
    \hlstd{min_diff_max} \hlkwb{<-} \hlkwd{which}\hlstd{(min_diff} \hlopt{==} \hlkwd{max}\hlstd{(min_diff))}
    \hlstd{supply_limit_place_holder[i]} \hlkwb{<-} \hlstd{min_diff_max[}\hlnum{1}\hlstd{]}

    \hlstd{abs_min_diff} \hlkwb{<-} \hlkwd{abs}\hlstd{(min_diff)}
    \hlstd{abs_min_diff} \hlkwb{<-} \hlkwd{which}\hlstd{(abs_min_diff} \hlopt{==} \hlkwd{min}\hlstd{(abs_min_diff))}
    \hlstd{supply_place_holder[i]} \hlkwb{<-} \hlstd{abs_min_diff[}\hlnum{1}\hlstd{]}
\hlstd{\}}
\end{alltt}
\end{kframe}
\end{knitrout}

%\begin(figure)
\begin{centering}
\begin{knitrout}
\definecolor{shadecolor}{rgb}{0.969, 0.969, 0.969}\color{fgcolor}\begin{figure}
\includegraphics[width=\maxwidth]{figure/unnamed-chunk-10-1} \caption{\label{fig:figs}(A) Unregulated and regulated transpiration with supply demand limit. Where curves intersect the x-axis indicate the predawn/soil matric potential. (B) Regulated vs unregulated transpiration. the differing curves represent the responses for the differing predawn/soil matric potentials in (A). (C) Loss of stomatal conductance NOTE I haven't worked out how Sperry is producing his Fig.4 in Sperry and Love (2015). The differing curves correspond to the differing predawn/soil matric potentials in (A). (D) percentage loss of conductance.}\label{fig:unnamed-chunk-10}
\end{figure}


\end{knitrout}
\end{centering}

%\caption{Plot of 1:10 and a bar plot beside it in a figure that is 4x6 inches}
%\end(figure)



%begin{centering}
%<<fig.width=16, echo=FALSE, fig.height=8, ,fig.cap="\\label{fig:figs}Regulated leaf matric potential against the demand defined matiric potential.">>==
%par(mfrow=c(1,1), mar=c(7,8,4,1))
%plot(psi_canopy, regulated_leaf_psi[,1], type="l", ylab="Delta P ", xlab="Sap pressure (-MPa)", cex.lab=2.0, cex.axis=2.0,lwd=2.5)
%text(4.5, 7.8,"(A)", cex=2)

%@
%\end{centering}

\end{document}











